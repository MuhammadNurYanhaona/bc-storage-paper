\section{Problem Model and Challenges}
\label{s-scope}
Any interaction between a storage administrator and a user of documents can only be of two types. A user either uploads or downloads a document to/from the storage system controlled by the administrator. The two participants of the interaction know each other by their blockchain identity (i.e., blockchain address) and any obligation of service between the administrator and the user is encoded in some blockchain smart contracts. In this arrangement, the parties can prove their identities to each other by simply being able to do blockchain transactions that update the relevant smart contracts. This works because any unauthenticated attempt to update a smart contract will be rejected by the blockchain network.      

Given blockchain smart contract languages are Turing complete, any complex access authorization logic can also be encoded in the same smart contracts that being used for participant authentication. Then the storage integration gateway configured to do transactions on behalf of the storage administrator can perform both authentication and authorization of user access by gleaning information from the blockchain. \textit{The challenge lies in, first, making the interaction between the user and the gateway secure and accountable for both parties; and, second, in making the access authorization logic generic to be applicable in a variety of use cases without rewriting smart contracts}.          

Using the blockchain network as the veritable source of information for governing user interactions with the storage system raises the concern that the version of the blockchain a user or the gateway observes by interacting with a selective subset of network peers may not be included in the canonical longest blockchain in the long run. Blockchain ledger reorgs \cite{reorg} can reverse the ledger of a peer to an alternate state at any time. Consequently, any storage access and allocation decision being made based on the old ledger state may later become disputable. \textit{Thus the third challenge for storage integration is to make the gateway responsive to ledger reorg without violating any service agreements}.         

Note that the involvement of blockchain network in document upload and download with external storages provides a natural mechanism for document integrity checking. During a document upload, a short and unique document signature can be generated from the document content and stored in the associated blockchain smart contract. During a download, the client application can recompute the signature from the downloaded content and match that with the signature found in the blockchain smart contract. If the two signatures do not match then the document has been modified or corrupted outside the guidance of the blockchain network.  

Subsequent sections describes how we solve the three core challenges for storage system integration with blockchain.
