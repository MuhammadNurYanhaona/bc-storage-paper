\section{Document Upload and Download Protocols}
\label{s-updown}
For the discussion of this section, we ignore the access permission and blockchain ledger reorganization related concerns. Subsequent sections address those issues. In addition, we assume that both document upload and download with the external storage involves blockchain payments. Protocols without payment can be easily derived from the described protocols by eliminating the steps related to payments. Finally, both protocols heavily use symmetric key cryptography in various steps for information and payment security. We refer the reader to \cite{1455525} for an introduction to cryptography and to \cite{Daemen99aesproposal:} for AES symmetric key cryptography in particular. 

\subsection{Quality Criteria for Upload/Download Protocol Design}
We decide on a set of behavioral characteristics for the document upload and download protocols. These characteristics are classified into two groups based on the expectation of the storage integration gateway and that of the user from the protocols. 
From the gateway's perspective the following characteristics are important:
\begin{itemize}
\item Accountability: all actions are traceable in the blockchain.
\item Independence from storage system failure: payment is collected only after all interactions with the external storage are done successfully.
\item Guaranteed Payment: the user can neither fool it to do unnecessary work nor withheld the payment after the work is complete.
\item Fault-tolerance: all local state information should be derivable from the blockchain ledger so that the gateway can restart easily after a failure or data corruption.  
\item Security from Malicious Miners: no mining nodes can snatch the payment intended for the gateway. 
\end{itemize}   
From the user's perspective, on the other hand, the following characteristics are desired from the protocols:
\begin{itemize}
\item Security: no one else can interrupt the storage session intended for the user. 
\item Payment Security: the user must be able to to get refunds if he/she does not get the proper service. 
\item Confidentiality: the underlying document cannot be retrieved by others (users or miners) using the audit trail of the protocol execution in the blockchain ledger.
\end{itemize} 
Propriety of both groups of behavioral characteristics is self-evident. Hence we do not elaborate on them further. During the protocol description, we discuss how different aspects of the protocols serve to achieve the mentioned characteristics. 

\subsection{Document Upload Protocol Description}
The document upload protocol is composed of three main phases:
\begin{enumerate}
\item Token Deposition: the {\it user} collects a payment token from the {\it gateway} and locks a document upload fee in the blockchain ledger for the {\it gateway} using another token.
\item Document Upload Processing: the {\it gateway} verifies that payment is locked and adequate then cooperates with the \textit{user} to upload the document in the {\textit external storage}.
\item Payment Finalization: the {\it user} verifies the document upload and unlocks the payment for the {\it gateway}.
\end{enumerate}
Each phase of the protocol involves multiple interaction steps among various system components. Figure~\ref{fig:up-proto} presents the sequence diagram of the protocol.
\tikzset{every picture/.append style={transform shape,scale=.60}}
\begin{figure}
  \begin{sequencediagram}
    \newthread{usr}{\it User}{}
    \newinst[1.1]{ac}{ \it Gateway}{}
    \newinst[1.1]{bc}{\it Blockchain Network}{}
    \newinst[1.1]{es}{\it Document Storage}{}

    \begin{callself}{usr}{\it Generate Document Key $D_k$}{}
    \end{callself}
    \begin{call}{usr}{\hspace{0.5cm} \it Get Initial Token}{ac}{\it M}
        \begin{callself}{ac}{\it Generate token M with generated key $K_1$}{}
        \end{callself}
    \end{call}
    \begin{callself}{usr}{\it Generate $K_2$ and prepare N}{}
    \end{callself}
    \begin{call}{usr}{\hspace{1cm} \it Upload Document Info and Lock Fee}{bc}{}
    \end{call}
    \begin{call}{usr}{\hspace{0.8cm} \it Upload Document}{ac}{}
        \begin{call}{ac}{\hspace{2.5cm} \it Verify doc size, fee and token M}{bc}{}
        \end{call}
        \begin{call}{ac}{\it Upload document}{es}{\it location Information}
        \end{call}
        \begin{call}{ac}{\hspace{2.5cm} \it Store encrypted location information}{bc}{}
        \end{call}
    \end{call}
    \begin{call}{usr}{\hspace{1.5cm} \it Check upload Status}{bc}{}
    \end{call}
    \begin{call}{usr}{\it Unlock payment with $K_2$}{bc}{}
        \begin{call}{bc}{\it Decrypt $N$ with $K_2$, match with M}{bc}{}
        \end{call}
    \end{call}

    \begin{call}{usr}{\hspace{0.5cm} \it Finalize upload}{ac}{}
        \begin{call}{ac}{\hspace{1.5cm} \it Check payment status}{bc}{}
        \end{call}
        \begin{call}{ac}{\hspace{1.5cm} \it Withdraw fee with $K_1$}{bc}{}
        \begin{callself}{bc}{\it Decrypt $M$ with $K_1$, match caller}{}
        \end{callself}
        \end{call}
    \end{call}
  \end{sequencediagram}
\caption{Sequence diagram of the document upload protocol}\label{fig:up-proto}
\end{figure}
We now describe these steps as parts of the three protocol phases.

\subsubsection{Token Deposition}
The {\it user} generates a document key $D_k$ with document information (document name $D_n$, document uploader blockchain address $DU_{addr}$ and document bearer contract address $DC_{addr}$) to identify the document.
\begin{equation}
\label{eq-u-1}
D_k = hash (D_n, DU_{addr}, DC_{addr}) 
\end{equation}
The {\it user} requests the {\it gateway} to generate a token for uploading a document with document size $D_{s}$, document hash $D_{h}$, signature of document hash $DH_{sig}$, $DU_{addr}$ and $D_{k}$. The {\it gateway} validates the input data from the {\it user} and verifies the signature using an already deployed smart contract (responsible to perform sign verification) from the blockchain node it is connected with. The {\it gateway} generates an AES symmetric key $K_{1}$ and encrypts its blockchain address $G_{addr}$ with the generated key and produces an payment token $M$. 
\begin{equation}
\label{eq-u-2}
M = Enc (G_{addr}, K_1)
\end{equation}
The {\it gateway} calculates the document upload fee according to the size of the document, stores $K_{1}$ for further use in next upload phase and returns the generated token $M$ and calculated $fee \; amount$ to the {\it user}.

The {\it user} generates another secret key $K_2$, encrypts the payment token $M$ with $K_2$ to produce an upload token $N$.
\begin{equation}
\label{eq-u-3}
N = Enc (M, K_2)
\end{equation}
The {\it user} performs a transaction to the blockchain with token $M$ and $N$ and the document information ($D_{n}$, $D_{h}$ and other document meta-data). During this operation, the calculated upload fee is also deposited to the blockchain smart contract.
 
\subsubsection{Document Upload Processing}
After the earlier transaction is mined, the user sends the document upload request to the {\it gateway} with the actual document and its meta-data. The {\it gateway} then retrieves the payment token $M^\prime$ from the blockchain and verifies that it contains the {\it gateway's} address by performing a decryption on token $M^\prime$ with key $K_1$ and checks that if the result is $G_{addr}$. The {\it gateway} also verifies the document size and deposited fee from the blockchain. If all verifications succeed, the {\it gateway} uploads the document to an external storage\footnote{If the storage system allows gateway defined transferable upload sessions for users then the user can do the document upload instead of the gateway.}, locally stores the document location information, and stores an encrypted version of the location information in the blockchain. The {\it user} can verify that the document  upload is successful by checking the blockchain. 

\subsubsection{Payment Finalization}
After a satisfactory verification of the outcome of upload, the user issues an unlock payment transaction with the user secret key $K_2^\prime$ to the blockchain. The blockchain verifies the key by decrypting $N$ with provided $K_2^\prime$ and matches with stored payment token $M$. If the verification succeed, the smart contract unlocks the payment to forward the amount to the receiver address encrypted as payment token $M$.
\begin{equation}
\label{eq-u-5}
M = Dec (N, K_2^\prime)
\end{equation}
The user requests the {\it gateway} to finalize the document upload procedure with document information and a signature. The {\it gateway} verifies the signature of the user and checks the payment unlock status from the blockchain. The {\it gateway} issues a transaction to collect the upload payment with its secret key $K_1$. The smart contract decrypts payment token $M$ with $K_1$ and matches the result with the caller address $C_{addr}$. If the address matches with decrypted result, smart contract transfers the payment to the caller  address.
\begin{equation}
\label{eq-u-6}
C_{addr} = Dec (M, K_1)
\end{equation}

\subsubsection*{Protocol Analysis}
Since it is cryptographically hard to produce an alternative address and key pair that satisfies Equation \ref{eq-u-6}, only the {\it gateway} can collect the payment. On the other hand, since payment is locked until someone supplies proper $K_2$ in the transaction evaluating Equation \ref{eq-u-5}, which is only known to the {\it user}, the {\it user} ensures that the gateway cannot collect the payment until the {\it user} verifies that the document is successfully uploaded. In addition, since the document's location information in the external storage is recorded by the {\it gateway} in the blockchain in an encrypted form, none but the {\it gateway} can interpret this information for future reference. This simultaneously ensures information security and gateway fault-tolerance. Finally, that user can issue the required transactions into the designated smart contracts indirectly ensures user authentication. 

Note that we did not address the possibility that the upload protocol terminates halfway in the execution for some machine or network failure. This issue can be tackled easily by associating an expiry time with the payment locking transaction. If the protocol fails before the {\it gateway} uploads the document in the external storage system, the user can withdraw the payment after the expiry time. If the failure happens after the document upload in the external storage, then the {\it gateway} can never collect the payment as the user is no longer there to unlock the payment for it. Hence, the {\it gateway} simply removes the document from the external storage in such cases.      

\subsection{Document Download Protocol}
The document download protocol being illustrated in Figure~\ref{fig:down-proto} consists of four phases.
\tikzset{every picture/.append style={transform shape,scale=1}}
\begin{figure}
  \label{seq:downloadProtocol}
   \begin{sequencediagram}
    \newthread{usr}{\it User}{}
    \newinst[1.2]{ac}{\it Gateway}{}
    \newinst[1.2]{bc}{\it Blockchain Network}{}
    \newinst[1.2]{es}{\it Document Storage}{}

    \begin{call}{usr}{\hspace{0.5cm} \it Get payment token}{ac}{\it  $T_p$, fee amount}
        \begin{callself}{ac}{\it Prepare $T_p$ with generated key $K_{a}$}{}
        \end{callself}
    \end{call}
    
    \begin{callself}{usr}{\it Generate $K_c$, $K_u$ and prepare $T_u$}{}
    \end{callself}
    \begin{call}{usr}{\it Deposit and lock fee}{bc}{}
    \end{call}
    
    \begin{call}{usr}{\hspace{0.5cm} \it Prepare session}{ac}{\hspace{0.1cm} \it $ENC_s$}
        \begin{call}{ac}{\hspace{1.5cm}\it Verify sign and payment}{bc}{\it $T_p$}
        \end{call}
        \begin{callself}{ac}{\it Verify $T_p$}{}
        \end{callself}
        \begin{call}{ac}{\it Generate session}{es}{$S$}
        \end{call}
        \begin{callself}{ac}{\it Generate $K_s$ and prepare $ENC_s$} {}
        \end{callself}
        \begin{callself}{ac}{\it Prepare Hash $H_s$ of $ENC_s$}{}
        \end{callself}
        \begin{call}{ac}{\hspace{0.3cm}\it $H_s$}{bc}{}
        \end{call}
    \end{call}

    \begin{call}{usr}{\hspace{1.0cm} \it verify $H_s$ is the hash of $ENC_s$}{bc}{}
    \end{call}
    \begin{call}{usr}{\hspace{1.0cm} \it Unlock payment with $K_u$, $D_k$, $U_{addr}$}{bc}{}
    \end{call}
    
    \begin{call}{usr}{\it Get session}{ac}{}
        \begin{call}{ac}{\hspace{3.0cm} \it Verify signature and payment status}{bc}{}
        \end{call}
        \begin{call}{ac}{\hspace{3.2cm} \it Withdraw payment with $K_a$, $ENC_{ks}$}{bc}{}
        \end{call}
    \end{call}

    \begin{call}{usr}{\hspace{0.8cm} \it Get encrypted session encryption key }{bc}{$ENC_{ks}$}
    \end{call}
    \begin{call}{usr}{\it Dec($ENC_s$, Dec($ENC_{ks}$, $K_c$))}{usr}{\it $S$}
    \end{call}
    \begin{call}{usr}{\it Download document}{es}{\it Document}
    \end{call}
  \end{sequencediagram}
\caption{Sequence diagram of the document download protocol}\label{fig:down-proto}
\end{figure}
\begin{enumerate}
\item Token Deposition: the {\it user} collects a payment token from the {\it gateway} and locks a download fee in the blockchain ledger for the {\it gateway} using another token.
\item Download Session Creation: the {\it gateway} generates a download session with the external storage and writes encrypted session information in the blockchain.
\item Payment Approval: the {\it user} checks information in the  blockchain, unlocks the download payment, and submits a fee transfer request for the {\it gateway}.
\item Document Download: the {\it user} collects the session encryption key from the {\it gateway}, regenerates the download session, and downloads the document from the external storage directly using that session.  
\end{enumerate}
The phases are described in more detail below. 

\subsubsection{Token Deposition}
The {\it user} requests the {\it gateway} to generate a document download payment token with the document key $D_k$, document hash $D_h$, {\it user's} blockchain address $U_{addr}$, and signature of document hash $DH_{sig}$. The {\it gateway} receives and validates the input data and verifies the signature $DH_{sig}$ using $D_{h}$ as message. The {\it gateway} then generates a payment secret key $K_a$ and encrypts its blockchain address $G_{addr}$ with the secret key and produces a payment token $T_{p}$. 
\begin{equation}
\label{eq-d-1}
T_p = Enc (AC_{addr}, K_a)
\end{equation}
The {\it gateway} also calculates the fee to download the document, stores the secret key $K_a$, and returns the payment token $T_p$ with the calculated download fee to the {\it user}.
After receiving the payment token, the {\it user} generates two symmetric keys $K_c$ and $K_u$. $K_c$ is the common conversation secret shared only with the {\it gateway}. The {\it user} generates an unlock token $T_u$ by encrypting $T_p$ with the key $K_u$.
\begin{equation}
\label{eq-d-2}
T_u = Enc (T_p, K_u)
\end{equation}
The {\it user} then issues a download payment transaction to the blockchain with two tokens $T_p$ and $T_u$, and some document related information that deposits the download fee in the blockchain smart contract.

\subsubsection{Download Session Creation}
After the download payment transaction is mined, the {\it user} issues a prepare download session request to the {\it gateway}. During this request, the {\it user} sends document bearer contract address $DC_{addr}$, user address $U_{addr}$, document key $D_k$, document hash $D_h$, signed document key $DK_{sig}$, and $K_c$. The {\it gateway} validates the input data and verifies the signature and payment status from the blockchain. If verification successful, the blockchain network returns the payment token $T_p^\prime$ found in the ledger to the {\it gateway}. The {\it gateway} then verifies $T_p^\prime$ by decrypting it with $K_a$ and matching the result with its own blockchain address. 
\begin{equation}
\label{eq-d-3}
G_{addr} = Dec (T_p^\prime, K_a)
\end{equation}
After successful address verification, the {\it gateway} retrieves the external storage information for the document and requests the {\it external storage} for a session $S$ to download the document. The {\it gateway} then generates a secret key $K_s$,  encrypts $S$ with $K_s$ and produces an encrypted session $ENC_s$. It also encrypts the document key with the common secret $K_c$ and produces an encrypted document key $ENC_{dk}$. 
\begin{equation}
\label{eq-d-4} 
ENC_s = Enc (S, K_s)
\end{equation}
\begin{equation}
\label{eq-d-5} 
ENC_{dk} = Enc (D_k, K_c)
\end{equation}
The {\it gateway} then prepares a hash, $H_s$, of $ENC_s$ and issues a transaction to the blockchain with $H_s$. The {\it gateway} locally stores all input data and returns $ENC_s$ to the {\it user}.

\subsubsection{Payment Approval}
The {\it user} verifies that the hash of $ENC_s$ is stored in the blockchain by the {\it gateway} then issues a transaction to unlock the payment with $K_u$, $D_k$ and $U_{addr}$. The smart contract unlocks the payment by decrypting the unlock token $T_u$ with the secret key $K_u$ provided by the {\it user} and matching the result with the payment token $T_p$. 
\begin{equation}
\label{eq-d-6} 
T_p = Dec (T_u, K_u)
\end{equation}
The {\it user} then issues a request to the {\it gateway} to get the download session with $D_k$, $DK_{sig}$ and $U_{addr}$. The {\it gateway} verifies the signature and checks payment unlock status from the  blockchain. Then it encrypts the session secret $K_s$ with common conversation secret $K_c$ and produces an encrypted session encryption key $ENC_{ks}$.
\begin{equation}
\label{eq-d-7} 
ENC_{ks} = Enc (K_s, K_c)
\end{equation}
The {\it gateway} issues a blockchain transaction to withdraw the payment with the $K_a$ and $ENC_{ks}$. The blockchain smart contract verifies the secret $K_a$ by decrypting the payment token $T_p$ with $K_a$ and matching the result with the caller address. If the decrypted result matches the caller address, blockchain transfers the download payment fee to the caller address. It also stores $ENC_{ks}$ with the download document information and notifies the {\it user}. The {\it user} then collects the encrypted session encryption key $ENC_{ks}$ from the blockchain.

\subsubsection{Document Download}
The {\it user} first decrypts the encrypted session encryption key $ENC_{ks}$ with the common secret $K_c$ and retrieves the session secret key $K_s$, then decrypts the encrypted session $ENC_s$ with $K_s$ to retrieve the original session $S$.
\begin{equation}
\label{eq-d-8} 
K_s = Dec (Enc_{ks}, K_c)
\end{equation}
\begin{equation}
\label{eq-d-9} 
S = Dec (Enc_s, K_s)
\end{equation}
Finally, the {\it user} downloads the document directly from the external storage using the session $S$.

\subsubsection*{Protocol Analysis}
Payment security for both the {\it user} and the {\it gateway} is ensured as it was in the upload protocol using a pair of encrypted tokens and proper sequencing of the payment related blockchain transactions. Likewise, all steps of the download protocol are also reflected on the blockchain smart contract. This ensures gateway accountability. Further, the {\it gateway} again collects the payment only after completing its interaction with the external storage to protect itself from any blame related to latter's failure. Finally, information related to the storage session  is recorded in the blockchain in encrypted form while the key $K_c$ for decrypting it is shared securely between the {\it user} and the {\it gateway}. Hence none but these two entities can use the session to get access to the document.

One aspect of the download protocol requires particular attention. This is related to unlocking of the download fee by the {\it user} before he/she can verify that the external storage session related information supplied by the {\it gateway} is accurate. Since the {\it user} is more likely to fail or be malicious, payment must be unlocked for the {\it gateway} before the {\it user} can decrypt the external storage session related information. However, this ordering opens the possibility that the {\it user} ends up paying for an invalid storage session or dispute its validity when it is actually valid. Therefore, a mechanism is required for automatically resolve any dispute related to external storage session in the blockchain smart contract. This is facilitated by the series of encryption related to the conversation secret $K_c$ and the storage session $S$.

To submit a claim about an invalid session, the {\it user} sends $K_c$ and $ENC_s$ in the blockchain smart contract. That the user has supplied the valid conversation secret is verified by performing the inverse of Equation \ref{eq-d-5} and matching the answer with the document key $D_k$. Since it is the {\it gateway} who did the original transaction, the {\it user} also establishes that the provided $K_c$ value was known to the {\it gateway}. Now the blockchain smart contract itself can compute Equation \ref{eq-d-8} and \ref{eq-d-9} to retrieve the session $S$. It then computes a hash of $S$ and checks if that matches $H_s$ which being stored in the blockchain by the {\it gateway}. If both hashes matches then the {\it user's} session invalidity claim is serious. Then the {\it gateway} maliciousness and external storage  malfunctioning should be investigated. Otherwise, the {\it user's} claim is ignored.          

